\section{Introduction}

Existing decentralised finance incentivises leaving interest-bearing products "idle" -- sitting in an account accruing interest. The majority of the world's population lives paycheck-to-paycheck and cannot afford to leave their money in an inaccessible account. This demographic is unable to lift itself out of poverty on the back of these instruments and is unable to participate in the financial system beyond simple banking.

Fluid Assets expose this demographic to decentralised finance. Fluidity Money's reward pool and yield generation mechanism operates similarly to a "no-loss lottery", pioneered by existing products, including PoolTogether \cite{pool}. Users exchange principal tokens into 1-to-1 backed Fluid Assets and that money is then lent on a yield-generating protocol. This cumulative yield makes up the reward pool that users are exposed to on each transfer of their wrapped (Fluid Asset) token. This reward mechanism incentivises taking Fluid Assets over their non-Fluid equivalent, as Fluid Assets can always be redeemed for their principal at no cost.

The system is modelled to be sound using the unique design of the Transfer Reward Function. Governance tokens are distributed through the use of a novel approach similar to liquidity mining titled Utility Mining. This unique approach to provisioning tokens incentivises adoption of the platform through the provision of liquidity. In the future, Fluidity will support guaranteed future yield by facilitating the sale of expected outcomes, similarly to Alchemix \cite{alchemix} and Pendle \cite{pendle}.

Fluidity Money's design necessitates careful and well-intentioned economics modelling. The platform should be resilient enough to withstand abuse from malicious actors while providing enough utility to be deemed useful. We examine Fluidity's design in the following.