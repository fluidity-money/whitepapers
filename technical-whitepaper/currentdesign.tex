
\section{Current Design}
\begin{center}
\includegraphics[width=11cm]{simplistic.png}
\end{center}
\subsection{Off-chain worker infrastructure}

Fluidity Money is powered by off-chain worker and randomness technologies. In the current iteration randomness servers ran magnanimously by the DAO observe transactions on-chain and generate randomness. Server randomness, signed, is then relayed via broadcast to a group of off-chain workers. The DAO servers source the current status of the APY and ATX of the underlying asset via an oracle that is supplied per transaction.

These off-chain workers are ran externally to Fluidity DAO. Workers receive randomness and compare it against their implementation of the Drawing Mechanism. Workers, upon realising that a winner was discovered, take the signed randomness, the block data, a Merkle-Patricia proof and relay it to the contract on-chain. The proof verifies the transaction was made in the last 256 blocks and the inputs are compared to determine if a winner should be paid out. If a winner was found, the contract pays the worker a small fee for their efforts and rewards the winner accordingly.

\subsection{Contract design}

Fluidity Ethereum contracts follow a standard factory pattern, with the following functions supported on the children:

\begin{lstlisting}

function erc20In(uint amount) public returns (uint);

function erc20Out(uint amount) public;

function rewardPoolAmount() public returns (uint);

function reward(
    uint blockIdx,
    uint256 idx,
    bytes memory proofBlob,
    bytes memory rngBlob,
    bytes memory rngSig
) public;

function verifyRewardReceipt(
    bytes32 blockHash,
    bytes memory proofBlob,
    address tokenAddr,
    bytes32 eventHash
) public pure returns (uint256, bytes32, bytes32);

function verifyRng(
    bytes memory rngRlp,
    bytes memory rngSig
) public view returns (
    bytes32 blockHash,
    uint txIndex,
    uint32 randomNumber,
    uint32 probability
);

\end{lstlisting}